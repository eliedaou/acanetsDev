% This syllabus template was created by:
% Brian R. Hall
% Assistant Professor, Champlain College
% www.brianrhall.net

% Document settings
\documentclass[11pt]{article}
\usepackage[margin=1in]{geometry}
\usepackage[pdftex]{graphicx}
\usepackage{multirow}
\usepackage{setspace}
\pagestyle{plain}
\setlength\parindent{0pt}

\begin{document}

% Course information
\begin{tabular}{ l l }
  \multirow{3}{*}{\includegraphics[height=1.25in]{UML-logo.png}} & \LARGE 16.480/552 \\\\
  & \LARGE Microprocessors II and Embedded System Design \\\\
  & \LARGE Wednesdays, 6:30pm - 9:20pm, Location: check ISIS \\\\
\end{tabular}
\vspace{10mm}

% Professor information
\begin{tabular}{ l l }
  \multirow{6}{*}{} & \large Prof. Yan Luo \\\\
  & \large Email: Yan\_Luo@uml.edu \\
  & \large Web: http://yanluo.github.io/ \\
  & \large Piazza: https://piazza.com/uml/fall2014/16480552/home \\
  & \large Office Location: Ball Hall 311 \\
  & \large Office Hours: Monday 1-3pm, Wednesday 4-5pm or by appointment \\
  & \large Tel: (978) 934-2592 \\
\end{tabular}
\vspace{5mm}
\begin{center} This syllabus may be subject to corrections and updates. \\
\end{center}

% Course details
\textbf {\large \\ Course Description:} 
This three-credit course provides a continuation of the study of microprocessors begun in 16.317. Topics include CPU architecture, memory interfaces and management, coprocessor interfaces, bus concepts, bus arbitration techniques, serial I/O devices, DMA and interrupt controlled devices. Focus will be placed on the design, construction, and testing of dedicated microprocessor systems (static and real-time). Hardware limitations of the single-chip system will be investigated along with microcontrollers, programming for small systems, interfacing, and communications, validating hardware and software, microprogramming of controller chips, and design methods and testing of embedded systems. Laboratories are directly related to microprocessor functions and embedded system designs. \\
\textbf {Prerequisite(s):} 16.311 - Electronics I Laboratory, 16.317 - Microprocessors Systems Design, 16.365 - Electronics I.

\textbf {Note(s):} All students enrolled in 16.480/552 Microprocessors Systems Design are required to have completed prerequisites or get permission from the Instructor. It is the student who is responsible for any adverse results such as being administratively withdrawn from the class or being ineligible for tuition refund due to the enforcement of these prerequisites.

\textbf {Credit Hours:} 3 \\

\textbf {\large Course Structure:} \\
There is a 3-hour class on Wednesday evening each week. The class will be in the format of lecture and laboratory.
\\

\textbf {\large Text(s):} \\
Barry B. Brey, {\em Intel Microprocessors}, 8th Ed. ISBN-10: 0135026458, ISBN-13: 978-0135026458

Peter Barry and Patrick Crowley, {\em Modern Embedded Computing: Designing Connected, Pervasive, Media-Rich Systems}, Morgan Kaufmann; 1 edition (February 10, 2012) ,ISBN-10: 0123914906, ISBN-13: 978-0123914903
\\

\textbf {\large Course Objectives:} \\
At the completion of this course, students will be able to:
\begin{enumerate} \itemsep-0.4em
  \item Gain an Understanding of Embedded Systems Design. 
  \item Become Capable of Evaluating and Implementing Memory System Organization,Decoding, and Timing.
\item Understand Different Communication Protocols and Interfaces.
\item Become Capable of Implementing Different Memory System Architectures.
\item Understand the Interaction of Hardware and Software in Embedded Systems
\item Understand emerging embedded system design technologies 
\end{enumerate}

% I recommend using \newpage here if necessary
\textbf {\large Grade Distribution:} \\
\hspace*{40mm}
\begin{tabular}{ l l }
Exam 1 & 20\% \\
Exam 2 & 20\% \\
Labs  & 60\% (four labs, 15\% each)\\
Total  & 100\%
\end{tabular} \\

Note: This course is double-numbered. Students enrolled in 16.480 are graded on a different scale from those in 16.552. Generally 60\% is the passing grade (D) for 16.480 students, 80\% is the passing grade (B) for 16.552 students.
\\
%\textbf {\large Letter Grade Distribution:} \\\\
%\hspace*{40mm}
%\begin{tabular}{ l l | l l }
%\textgreater= 93.00 & A & 73.00 - 76.99 & C \\
%90.00 - 92.99 & A-  & 70.00 - 72.99 & C- \\
%87.00 - 89.99 & B+  & 67.00 - 69.99 & D+ \\
%83.00 - 86.99 & B  & 63.00 - 66.99 & D \\
%80.00 - 82.99 & B-  & 60.00 - 62.99 & D- \\
%77.00 - 79.99 & C+  & \textless= 59.99 & F \\
%\end{tabular} \\

% Course Policies. These are just examples, modify to your liking.
\textbf {\large Course Policies:}
\begin{itemize}
	\item \textbf {General}
		\begin{itemize}
			%\item Computers are not to be used unless instructed to do so.
			\item Exams are closed book, closed notes unless instructed otherwise.
			\item \textbf {No makeup exams will be given unless proper documents are presented (such as doctor's notes or police reports).}
		\end{itemize}
	\item \textbf {Labs and Reports}
		\begin{itemize}
			\item Students are expected to work in teams of up to four people. However, {\bf each student is expected to complete and submit his/her own lab report.} The student must clearly establish authorship of a work in the report.

\textbf{Copying} reports from others is an act of \textbf{plagiarism}, which is a serious offense and \textbf{all involved parties will be penalized according to the Academic Integrity Policy}.
			\item \textbf{Penalty will be applied to late turn-ins: 20\% deduction per one late day.}.
		\end{itemize}
	\item \textbf{Attendance and Absences}
		\begin{itemize}
			\item Attendance is expected and may be taken each class. 

			\item Students are responsible for all missed work, regardless of the reason for absence. It is also the absentee's responsibility to get all missing notes or materials. 
		\end{itemize}
\end{itemize}

% College Policies
\textbf {\large Academic Integrity Policy :} 
% This should be specific to your instituition, an example is provided.

The university has a responsibility to promote academic honesty and integrity and to develop procedures to deal effectively with instances of academic dishonesty. Students are responsible for the honest completion and representation of their work, for the appropriate citation of sources, and for respect of others’ academic endeavors. Academic dishonesty is prohibited in all programs of the university.

Detailes are at http://www.uml.edu/Catalog/Undergraduate/Policies/Academic-Integrity.aspx


\newpage

% Course Outline
\textbf {\large Tentative Course Outline}:

The weekly coverage might change as it depends on the progress of the class.  However, you must check course Piazza forum for updates.

\begin{table}[h!]
\normalsize % The size of the table text can be changed depending on content. Remove if desired.
\begin{tabular}{ | c | c | c | }
\hline
\textbf{Week} & \textbf{Content} & \textbf{Labs Due Dates}\\
\hline
Week 1 & \begin{minipage}{.60\textwidth}
\begin{itemize} \itemsep-0.4em
	\vspace{1mm}
	\item Introduction to Embedded System Design
	\item Lab tutorial: PIC microcontroller and IDE
	\vspace{1mm}
\end{itemize}
\end{minipage} 
& teams formed
\\
\hline
Week 2 & \begin{minipage}{.60\textwidth}
\begin{itemize} \itemsep-0.4em
	\vspace{1mm}
	\item Sensors and Data Acquisition
	\item Lab 1 released: PIC-controlled sensing
	\vspace{1mm}
\end{itemize}
\end{minipage} 
& \\
\hline
Week 3 & \begin{minipage}{.60\textwidth}
\begin{itemize} \itemsep-0.4em
	\vspace{1mm}
	\item Lab Session
	\item (no lecture)
	\vspace{1mm}
\end{itemize}
\end{minipage} 
& \\
\hline
Week 4 & \begin{minipage}{.60\textwidth}
\begin{itemize} \itemsep-0.4em
	\vspace{1mm}
	\item Intro to x86 architecture, memory interfacing
        \item Lab tutorial: Intel Galileo
	\vspace{1mm}
\end{itemize}
\end{minipage} 
& Lab 1 due\\
\hline
Week 5 & \begin{minipage}{.60\textwidth}
\begin{itemize} \itemsep-0.4em
	\vspace{1mm}
	\item Interfaces and Buses
	\item Lab 2 released : PIC+Galileo: sensor control and data acquisition
	\vspace{1mm}
\end{itemize}
\end{minipage} 
&  \\
\hline
Week 6 & \begin{minipage}{.60\textwidth}
\begin{itemize} \itemsep-0.4em
	\vspace{1mm}
	\item Exam 1 + Lab session
	\item (no lecture)
	\vspace{1mm}
\end{itemize}
\end{minipage} 
& \\
\hline
Week 7 & \begin{minipage}{.60\textwidth}
\begin{itemize} \itemsep-0.4em
	\vspace{1mm}
	\item PCI, DMA, interrupts, storage devices
	\vspace{1mm}
\end{itemize}
\end{minipage} 
& Lab 2 due\\
\hline
Week 8 & \begin{minipage}{.60\textwidth}
\begin{itemize} \itemsep-0.4em
	\vspace{1mm}
	\item Linux device driver (1)
        \item Lab 3 released: Linux device driver 
	\vspace{1mm}
\end{itemize}
\end{minipage} 
& \\
\hline
Week 9 & \begin{minipage}{.60\textwidth}
\begin{itemize} \itemsep-0.4em
	\vspace{1mm}
	\item Linux device driver (2)
	\item Lab tutorial
	\vspace{1mm}
\end{itemize}
\end{minipage} 
& \\
\hline
Week 10 & \begin{minipage}{.60\textwidth}
\begin{itemize} \itemsep-0.4em
	\vspace{1mm}
	\item Lab session
	\item (no lecture)
	\vspace{1mm}
\end{itemize}
\end{minipage} 
& \\
\hline
Week 11 & \begin{minipage}{.60\textwidth}
\begin{itemize} \itemsep-0.4em
	\vspace{1mm}
	\item Parallel programming and Multithreading 
	\item Lab 4 released: Multithreading and networking
	\vspace{1mm}
\end{itemize}
\end{minipage} 
& Lab 3 Due\\
\hline
Week 12 & \begin{minipage}{.60\textwidth}
\begin{itemize} \itemsep-0.4em
	\vspace{1mm}
	\item networking of embedded devices
	\item demo
	\vspace{1mm}
\end{itemize}
\end{minipage}
& \\
\hline
Week 13 & \begin{minipage}{.60\textwidth}
\begin{itemize} \itemsep-0.4em
	\vspace{1mm}
	\item Thursday schedule 
	\item (no class)
	\vspace{1mm}
\end{itemize}
\end{minipage} & \\
\hline
Week 14 & \begin{minipage}{.60\textwidth}
\begin{itemize} \itemsep-0.4em
	\vspace{1mm}
	\item Emerging Embedded System Technologies (Android, IoT, etc.)
	\vspace{1mm}
\end{itemize}
\end{minipage} 
& \\
\hline
Week 15 & \begin{minipage}{.60\textwidth}
\begin{itemize} \itemsep-0.4em
	\vspace{1mm}
	\item Exam 2 and lab session
	\vspace{1mm}
\end{itemize}
\end{minipage} 
& Lab 4 due\\
\hline
\end{tabular} 
\end{table}

\end{document}



